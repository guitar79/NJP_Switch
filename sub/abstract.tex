\maketitle  % command to print the title page with above variables

\setcounter{page}{1}
%---------------------------------------------------------------------
%                  영문 초록을 입력하시오
%---------------------------------------------------------------------
\begin{abstracts}     %this creates the heading for the abstract page
	\addcontentsline{toc}{section}{Abstract}  %%% TOC에 표시
	\noindent{
	Seasonal variability of chlorophyll-a concentration has been analyzed by using SeaWiFS (Sea-viewing Wide Field-of-view Sensor) LAC (Local Area Coverage) and GAC (Global Area Coverage) data in the East Sea from 2003 to 2006. Both data presented a similarity with two peaks in spring (April) and fall (November) for a year. However, the monthly mean value of chlorophyll-a concentration was different, the maximum and minimum value of LAC and GAC data each being 1.61 $\rm mg~m^{-3}$, 0.28 $\rm mg~m^{-3}$, and 1.72 $\rm mg~m^{-3}$, 0.32 $\rm mg~m^{-3}$. The difference between the two data was less than 0.21 $\rm mg~m^{-3}$ every month, but it was not a small value. It could be inferred that the spatial resolution of a data must be considered more deeply for research on the chlorophyll-a concentration variability.  
	}
\end{abstracts}

%---------------------------------------------------------------------
%                  국문 초록을 입력하시오
%---------------------------------------------------------------------
\begin{abstractskor}        %this creates the heading for the abstract page
	\addcontentsline{toc}{section}{초록}  %%% TOC에 표시
	\noindent{
SeaWiFS (Sea-viewing Wide Field-of-view Sensor) 의 LAC (Local Area Coverage) 와 GAC (Global Area Coverage) 자료를 이용하여 2003년 부터 2006년 까지 동해의 클로로필-a 계절적 변동을 분석하였다. 두 자료 모두 봄철 (4월)과 가을철 (11월)에 월평균 농도가 높게 산출되어 전체적인 경향은 비슷하게 나타났다. 그러나 LAC와 GAC 자료로 산출한 chlorophyll-a 월평균 농도의 최대값, 최소값은 각각 1.61 $\rm mg~m^{-3}$, 0.28 $\rm mg~m^{-3}$ 과 1.72 $\rm mg~m^{-3}$, 0.32 $\rm mg~m^{-3}$ 으로 차이가 나타났다. 두 데이터의 월평균 값의 차이는 항상 0.21 $\rm mg~m^{-3}$ 보다 작았지만, 이 값은 무시할 수 없는 값이었다. 아러한 차이를 통하여 클로로필-a 농도 변동을 분석할때 데이터의 공간 해상도는 깊게 고려할 필요가 있음을 확인하였다. 	
	}
\end{abstractskor}

%----------------------------------------------
%   Table of Contents (자동 작성됨)
%----------------------------------------------
\cleardoublepage
\addcontentsline{toc}{section}{Contents}
\setcounter{secnumdepth}{3} % organisational level that receives a numbers
\setcounter{tocdepth}{3}    % print table of contents for level 3
\baselineskip=2.2em
\tableofcontents


%----------------------------------------------
%     List of Figures/Tables (자동 작성됨)
%----------------------------------------------
\cleardoublepage
\clearpage
\listoftables
% 표 목록과 캡션을 출력한다. 만약 논문에 표가 없다면 이 위 줄의 맨 앞에 
% `%' 기호를 넣어서 주석 처리한다.

\cleardoublepage
\clearpage
\listoffigures
% 그림 목록과 캡션을 출력한다. 만약 논문에 그림이 없다면 이 위 줄의 맨 앞에 
% `%' 기호를 넣어서 주석 처리한다.

\cleardoublepage
\clearpage
%\listofequations
% 그림 목록과 캡션을 출력한다. 만약 논문에 그림이 없다면 이 위 줄의 맨 앞에 
% `%' 기호를 넣어서 주석 처리한다.


\cleardoublepage
\clearpage
\renewcommand{\thepage}{\arabic{page}}
\setcounter{page}{1}