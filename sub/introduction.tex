\section{Introduction}

Ocean color remote sensing allows the indirect measurement of various matters in the ocean. The Coastal Zone Color Scanner (CZCS) is the first ocean color sensor, operated from 1978 to 1986 and the Sea-viewing Wide Field-of-view Sensor (SeaWiFS) has observed global ocean color distributions for about a decade, from 1998 to 2010, and has provided the scientific community with abundant information for a variety of oceanic application research \cite{kyung2013characteristics, hooker1992An}. Ocean color remote sensing is executed by collecting several bands of reflected light from the ocean and calculating the collected data to amount of matters in water using the algorithms developed through. Such data collected through this method is called ocean color data. Ocean color data is important in understanding the temporal and spatial distribution of algae \cite{kimhc2016surface}.

Chlorophyll-a concentration can be calculated using ocean color remote sensing. Since chlorophyll-a is a pigment that is required in photosynthesis, it is also found in algae. Additionally, chlorophyll-a concentration is used as a determinant for the amount of algae in ocean \cite{o2000ocean}. 

There have been several studies about chlorophyll-a concentration in the East Sea using ocean color remote sensing. The chlorophyll-a concentration was found to be related with wind, sea surface temperature, and many other variables of the ocean. Spring bloom and fall bloom were observed, showing a seasonal pattern \cite{yamada2004seasonal}. SeaWiFS satellite is often used for studying the chlorophyll-a concentration in the East Sea. It was found out that some of the area had abnormally higher concentration of chlorophyll-a compared to the areas nearby. These areas are called speckles. Speckles are errors made by SeaWiFS, so there had been studies to correct them. \cite{chae2009characteristics}. 

SeaWiFS creates two types of data with different spatial resolutions, Local Area Coverage (LAC) data and Global Area Coverage (GAC) data. LAC data is created with full-resolution (1.1 $\rm km$) for local area by combining the continuously recorded images. On the other hand, GAC data is created with low resolution (4.5 $\rm km$) subsampled from the full-resolution data with every fourth pixel to show a global area \cite{Seawifsres}. 

Although many scientists used SeaWiFS LAC data and GAC data for research, they did not found which data is more suitable. Which data is more accurate There is also no proof that applying the same algorithms gives the same accurate results. Thus, the difference between the LAC and GAC data should be discussed. 

The goals of this research are the followings. First, observing the chlorophyll-a concentration variability in the East Sea from 2003 to 2006 using SeaWiFS ocean color data. Second, comparing the chlorophyll-a concentration data of LAC and GAC to find the effect of spatial resolution.
