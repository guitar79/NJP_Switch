%-----------------------------------------------------
%   감사의 글
%-----------------------------------------------------
\begin{acknowledgements}
\addcontentsline{toc}{section}{Acknowledgements}  %%% TOC에 표시

%$# 아래 내용 골자로 해서 살을 붙이고, 에피소드 넣어서 작성하도록

%$# 이 논문은 미국 미시시피 주립대학에서 진행되었던 Overseas Research and education Program(ORP)의 지원을 받아 시작할 수 있었습니다. ORP를 담당했던 미시시피 주립대학의 Dr. Dash 교수님께 다시한번 감사드리며, Python coding, Ocean color data order, batch processing 등을 지도해 준 Madhur Dacoda에게도 감사드립니다.
 I would first like to thank Gyeonggi Science High school for the gifted for allowing us to study remote sensing through the Overseas Research and education Program(ORP) in Mississippi State University. Thanks for Dr. Padmanava Dash and Madhur Dacoda for teaching us about ocean color data with python coding and batch processing. The basic concepts and the methods that you have though us was really useful to process the data. The ORP program was more helpful with the support of Youngwoo Cho in Mississippi State University and my friend Jong Min, Choi and our teacher Kiehyun Park. The hardships we overcame while learning python was really meaningful.
 
 Lastly, I would like to thank Prof. Kyung Ae Park and Prof. Hak-sung Kim for revising this thesis. The overall comments and the specific corrections helped me to create a more professional thesis.  
%$# ORP 동안 도움을 주신 미시시피 주립대학의 Youngwoo Cho 선생님과 함께 참여한 경기과학고등학교의 친구 Jong Min, Choi, Kiehyun Park 선생님께도 감사를 드립니다.

%$# ORP 동안 도움을 주신 미시시피 주립대학의 Youngwoo Cho 선생님과 함께 참여한 경기과학고등학교의 친구 Jong Min, Choi, Kiehyun Park 선생님께도 감사를 드립니다. 감

%$# 경기과학고에 입학하여 무사히 졸업논문을 통과할 수 있도록 도와주신 아무개, 아무개, 아무개 선생님을 비롯하여 모든 선생님, 교직원 분들께 감사드리며, 

%$# 나를 낳아주시고 길러주신 사랑하는 아버지, 어머니, 형제, 자매에 감사의 말씀을 전합니다.



%%마지막으로 부족했던 이 논문의 완성도를 높일 수 있도록 심사위원장을 맡아주신 박경애 교수님, 그리고 심사위원을 맡아주신 김학성 교수님, 박기현 선생님께 다시한번 감사의 말씀을 전하고 싶습니다.


\end{acknowledgements}

%-----------------------------------------------------
%   연구활동 
%-----------------------------------------------------
\begin{researches}
\addcontentsline{toc}{section}{연구활동}  %%% TOC에 표시
\begin{itemize}
\item{2015. 03. 01 \~{} 2016. 2. 28. 경기과학고등학교 기초 R\&E (연구대상, 우수상) : Analysis on the Physical Characterisitics of Plates by Modeling the Earthquakes occurred in the Subduction Zone.}
\item{2015. 9. 20. International Earth Science Olympiad at Pocas de Caldas, Brazil (Silver medal award).}
\item{2015. 9. 10. 한국지구과학회 포스터 발표 (우수상) : 진원 분포 모델링을 통한 해구에서의 판의 물리적 특성 분석.}
\item{2015. 11. 23. STEAM R\&E 페스티벌 (우수상) : 전향력을 느끼고 체험하는 놀이기구 제작에 대한 연구.}
\item{2016. 2. 25. 과학영재창의연구학술발표대회 (우수상) : 지진 발생에서 진원 분포 모델링을 통한 해구에서의 판의 물리적 특성 분석에 대한 연구.}
\item{2016. 03. 01 \~{} 2017. 2. 28. 과학영재학교 운영 지원사업 I\&D 연구활동 : 자동기상관측 및 실시간 대기질 모니터링 시스템 구축과 활용.}
\item{2016. 03. 01 \~{} 2017. 2. 28. 경기과학고등학교 심화 R\&E : $  \rm Fe_{2}O_{3}$의 자성을 이용한 염료감응형 태양전지의 효율 상승.}
\item{2016. 06. 10. 제 62회 경기도과학전람회 (장려상) : 진원 분포 모델링을 통한 해구에서의 판의 물리적 특성 분석 및 해구 섭입 모형 제작.}
\item{2017. 2. 04. \~{} 2017. 2. 21. Overseas Research and education Program at Mississippi State University in U.S. : Water Quality Monitoring Using Remote Sensing data (GOCI, SeaWIFS, OCM) : Digital Image Processing.}
\item{2017. 4. 07. 한국지구과학회 포스터 발표 : 해색 자료를 이용한 동해에서의 Chlorophyll a 월평균 농도 산출 및 분석.}
\item{2017. 4. 20. 제 63회 경기도과학전람회 과학고 예선 (특상) : 해색자료를 이용한 동해에서의 Chlorophyll a 농도 분석 및 풍속과의 관계 파악.}
\item{2017. 5. 26. 제 63회 경기도과학전람회 : SeaWiFS 해색 자료를 이용한 동해에서의 Chlorophyll-a 농도 분석 : LAC 자료와 GAC 자료의 비교.}
\end{itemize}
\end{researches}