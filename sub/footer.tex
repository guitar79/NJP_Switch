%-----------------------------------------------------
%   감사의 글
%-----------------------------------------------------
\begin{acknowledgements}
\addcontentsline{toc}{section}{감사의 글}  %%% TOC에 표시

%$# 아래 내용 골자로 해서 살을 붙이고, 에피소드 넣어서 작성하도록

%$# 이 논문은 미국 미시시피 주립대학에서 진행되었던 Overseas Research and education Program(ORP)의 지원을 받아 시작할 수 있었습니다. ORP를 담당했던 미시시피 주립대학의 Dr. Dash 교수님께 다시한번 감사드리며, Python coding, Ocean color data order, batch processing 등을 지도해 준 Madhur Dacoda에게도 감사드립니다.

%$# ORP 동안 도움을 주신 미시시피 주립대학의 Youngwoo Cho 선생님과 함께 참여한 경기과학고등학교의 친구 Jong Min, Choi, Kiehyun Park 선생님께도 감사를 드립니다.

%$# ORP 동안 도움을 주신 미시시피 주립대학의 Youngwoo Cho 선생님과 함께 참여한 경기과학고등학교의 친구 Jong Min, Choi, Kiehyun Park 선생님께도 감사를 드립니다.

%$# 경기과학고에 입학하여 무사히 졸업논문을 통과할 수 있도록 도와주신 아무개, 아무개, 아무개 선생님을 비롯하여 모든 선생님, 교직원 분들께 감사드리며, 

%$# 나를 낳아주시고 길러주신 사랑하는 아버지, 어머니, 형제, 자매에 감사의 말씀을 전합니다.

Dr. Dash and Madhur in Mississippi State University for Park, Kie Hyun Gyeonggi Science High school 

%%마지막으로 부족했던 이 논문의 완성도를 높일 수 있도록 심사위원장을 맡아주신 박경애 교수님, 그리고 심사위원을 맡아주신 김학성 교수님, 박기현 선생님께 다시한번 감사의 말씀을 전하고 싶습니다.
I would like to thank Prof. Park and Prof. Kim for revising this thesis. 

\end{acknowledgements}

%-----------------------------------------------------
%   연구활동 
%-----------------------------------------------------
\begin{researches}
\addcontentsline{toc}{section}{연구활동}  %%% TOC에 표시
\begin{itemize}
\item{2014. 8.    . Participated in the International Earth Science Olympiad at ooo(city) Brazil. - (Silver medal award).}
\item{2017. 2.    . Participated in the Overseas Research and education Program at Mississippi state university U.S. : Water Quality Monitoring Using Remote Sensing data (GOCI, SeaWIFS, OCM) : Digital Image Processing. }
\item{2017. 4.    . 춘계지구과학회 포스터 발표 : 제목 }
\item{2017. 4.    . 경기도과학전람회 과학고 예선: 제목 }
\item{2017. 5.    . 경기도과학전람회: 제목 }
\end{itemize}
\end{researches}